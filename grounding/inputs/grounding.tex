\begin{Uebung}{(Grounding)}
For the following logic programs~$\Pi$ and sets of facts~$I$:
\begin{itemize}
\item Find the dependency graph $G_{\Pi}$,
      the positive dependency graph $G_{\Pi}^+$,
      and one topological order $L_{\Pi}$.
\item %For every rule $r$ occurring in a component $C$ of 
      Given $L_{\Pi}$,
      determine the set of atoms $R_r$ for every rule $r \in P$.\footnote{%
      Atom $a \in R_r$ iff (i) $a \in \nbody{r}$ and (ii)
      the predicate of $a$ occurs in the head of some rule
      that belongs either to the component $C$ of $L_{\Pi}$ such that $r \in C$, or
      to some other component that occurs after $C$ in $L_{\Pi}$.
      Note that if $r$ is not involved in a cycle containing negative edges, then $R_r$ is empty.
      On the other hand, if $r$ is involved in a cycle containing negative edges, then $R_r$ can either be empty or not.}
\item Find the ground instantiation of~$\Pi$ and~$I$.\footnote{%
      You can check this with \texttt{clingo} 
      using the command \texttt{clingo --text program.lp facts.lp}.}
      Start by initializing the sets of true atoms~$F$ and possible atoms $D$ to~$I$.
      Then, ground successively the components of $L_{\Pi}$ applying on-the-fly simplifications.
      For this, use the sets $F$, $D$ and $R_r$,
      and update~$F$ and~$D$ after grounding each component.
\end{itemize}

%
\begin{UList}
\item
\(
%\begin{array}{l@{\hspace{4mm}}l}
  \Pi 
  = 
  \left\{
  \begin{array}{r@{}c@{}l}
    select(X) & {} \leftarrow {} & vertex(X),\naf{nselect(X)} \\
    nselect(X) & {} \leftarrow {} & vertex(X),\naf{select(X)} \\
      & {} \leftarrow {} & edge(X,Y), select(X), select(Y)
  \end{array}
  \right\}
\)

% &
\(
  I 
  = 
  \left\{
  \begin{array}{r@{}c}
    vertex(1..3) & {} \leftarrow \\
    edge(1,2) & {} \leftarrow \\
    edge(2,3) & {} \leftarrow \\
    edge(3,4) & {} \leftarrow \\
    select(1) & {} \leftarrow
  \end{array}
  \right\}
%\end{array}
\)
%
\vspace{5mm}
\item
\(
%\begin{array}{l@{\hspace{4mm}}l}
  \Pi\text{\footnotemark}
  = 
  \left\{
  \begin{array}{r@{}c@{}l}
    select(X) & {} \leftarrow {} & element(S,X), \naf{nselect(X)} \\
    nselect(X) & {} \leftarrow {} & element(S,X), \naf{select(X)} \\
    & {} \leftarrow {} & set(S), element(S,X1), select(X1), \\
    & {}            {} & element(S,X2), select(X2), X1 \neq X2\\
    hit(S) & {} \leftarrow {} & element(S,X), select(X) \\
    & {} \leftarrow {} & set(S), \naf{hit(S)}
  \end{array}
  \right\}
\) % &

\(
  I 
  = 
  \left\{
  \begin{array}{r@{}c}
    set(1..3) & {} \leftarrow \\
    element(1,a) & {} \leftarrow \\
    element(2,b) & {} \leftarrow \\
    element(2,c) & {} \leftarrow \\
    element(3,b) & {} \leftarrow \\
    nselect(a) & {} \leftarrow
  \end{array}
  \right\}
% \end{array}
\)
\footnotetext{Comparison literals are completely evaluated during grounding.}
%
\newpage 
\item
\(
\begin{array}{l@{\hspace{2mm}}l}
  \Pi\text{\footnotemark}
  = 
  \left\{
  \begin{array}{r@{}c@{}l}
    \{ select(X) \}& {} \leftarrow {} & vertex(X) \\
    covered(X,Y) & {} \leftarrow {} & edge(X,Y), select(X) \\
    covered(X,Y) & {} \leftarrow {} & edge(X,Y), select(Y) \\
    & {} \leftarrow {} & edge(X,Y), \naf{covered(X,Y)}
  \end{array}
  \right\}
&
  I 
  = 
  \left\{
  \begin{array}{r@{}c}
    vertex(1..4) & {} \leftarrow \\
    edge(1,2) & {} \leftarrow \\
    edge(2,3) & {} \leftarrow \\
    edge(3,1) & {} \leftarrow \\
    select(1) & {} \leftarrow
  \end{array}
  \right\}
\end{array}
\)
\footnotetext{Choice rules are handled analogously to normal rules,
except that they never contribute true atoms to the set $F$.}
%\phantom{A\footnote{x}}
%
\vspace{5mm}
\item
\(
\begin{array}{l@{\hspace{4mm}}l}
  \Pi 
  = 
  \left\{
  \begin{array}{r@{}c@{}l}
    dominated(X) & {} \leftarrow {} & select(X) \\
    dominated(X) & {} \leftarrow {} & edge(X,Y), select(Y) \\
    dominated(X) & {} \leftarrow {} & edge(Y,X), select(Y) \\
    incorrect & {} \leftarrow {} & vertex(X), \naf{dominated(X)} \\
    correct & {} \leftarrow {} & \naf{incorrect}
  \end{array}
  \right\}
&
  I 
  = 
  \left\{
  \begin{array}{r@{}c}
    vertex(1..4) & {} \leftarrow \\
    edge(1,2) & {} \leftarrow \\
    edge(2,3) & {} \leftarrow \\
    edge(3,4) & {} \leftarrow \\
    select(1) & {} \leftarrow  \\
    select(4) & {} \leftarrow
  \end{array}
  \right\}
\end{array}
\)
%
\vspace{5mm}
\item
\(
% \begin{array}{l@{\hspace{4mm}}l}
  \Pi 
  = 
  \left\{
  \begin{array}{r@{}c@{}l}
    holds(V1,A) & {} \leftarrow {} & atom(A), opp(V1,V2), \\
                & {}            {} & \naf{holds(V2,A)} \\
    satisfied(C) & {} \leftarrow {} & literal(C,V,A), holds(V,A) \\
    & {} \leftarrow {} & clause(C), \naf{satisfied(C)}
  \end{array}
  \right\}
% &
\)

\(
  I 
  = 
  \left\{
  \begin{array}{r@{}c}
    atom(a;b;c) & {} \leftarrow \\
    clause(1..3) & {} \leftarrow \\
    literal(1,pos,a) & {} \leftarrow \\
    literal(2,neg,a) & {} \leftarrow \\
    literal(2,pos,b) & {} \leftarrow \\
    literal(3,neg,b) & {} \leftarrow \\
    literal(3,pos,c) & {} \leftarrow \\
    opp(neg,pos) & {} \leftarrow \\
    opp(pos,neg) & {} \leftarrow \\
    holds(pos,a) & {} \leftarrow  \\
  \end{array}
  \right\}
% \end{array}
\)
%
\end{UList}
\end{Uebung}

\begin{Loesung}%
\textnormal{%
\begin{itemize}
\item
For each program $\Pi$,
we refer by $r_1$ to the first written rule,
by $r_2$ to the second written rule, and so on\ldots
\item
For every program $\Pi$ and set of facts $I$,
the ground instantiation of $\Pi$ and $I$
is the result of joining the ground instantiations of all components in $L_{\Pi}$.
\end{itemize}
}
\begin{UList}%
%
% a
\item 
\begin{description}
%
\item[Dependency Graph:] ~\\
\(
  G_{\Pi} = \big( \Pi, \{(r_1,r_2), (r_2,r_1), (r_1,r_3)\} \big)
\)
%
\item[Positive Dependency Graph:] ~\\
\(
  G_{\Pi}^+ = \big( \Pi, \{(r_1,r_3)\} \big)
\)
%
\item[Topological orders:] ~\\
\textnormal{%
The topological orders are
\begin{itemize}
\item 
$(\{r_1\},\{r_2\},\{r_3\})$, and
\item 
$(\{r_2\},\{r_1\},\{r_3\})$.
\end{itemize}
In the exercise we only need to find one,
and in our case we choose \\ $L_{\Pi}=(\{r_1\},\{r_2\},\{r_3\})$.
}
%
\item[Sets $R_r$ for $r \in \Pi$:] ~\\
\textnormal{%
$R_{r_1}=\{{nselect(X)}\}$, $R_{r_2}=\emptyset$, and $R_{r_3}=\emptyset$.
}
%
\item[Grounding:] ~\
\textnormal{%
\begin{enumerate}
  \setcounter{enumi}{-1}
  \item Initialization: $F = I$, $D = I$
  \item Ground instantiation of component $\{r_1\}$:
  \( \left\{
  \begin{array}{r@{}c@{}l}
    select(2) & {} \leftarrow {} & \naf{nselect(2)} \\
    select(3) & {} \leftarrow {} & \naf{nselect(3)}
  \end{array}
  \right\}
  \)
  \\
  $F = I$ 
  \\ 
  $D = I \cup \{ select(2), select(3) \}$
  \item Ground instantiation of component $\{r_2\}$:
  \( \left\{
  \begin{array}{r@{}c@{}l}
    nselect(2) & {} \leftarrow {} & \naf{select(2)} \\
    nselect(3) & {} \leftarrow {} & \naf{select(3)}
  \end{array}
  \right\}
  \)
  \\
  $F = I$ 
  \\ 
  $D = I \cup \{ select(2), select(3), nselect(2), nselect(3) \}$
  \item Ground instantiation of component $\{r_3\}$:
  \( \left\{
  \begin{array}{r@{}c@{}l}
    & {} \leftarrow {} & select(2) \\
    & {} \leftarrow {} & select(2), select(3)
  \end{array}
  \right\}
  \)
  \\
  $F = I$ 
  \\ 
  $D = I \cup \{ select(2), select(3), nselect(2), nselect(3) \}$
\end{enumerate}
}
%
\end{description}
%
% b
\item 
\begin{description}
%
\item[Dependency Graph:] ~\\
\(
  G_{\Pi} = \big( \Pi, \{(r_1,r_2), (r_2,r_1), (r_1,r_3), (r_1,r_4), (r_4,r_5)\} \big)
\)
%
\item[Positive Dependency Graph:] ~\\
\(
  G_{\Pi}^+ = \big( \Pi, \{(r_1,r_3), (r_1,r_4)\} \big)
\)
%
\item[Topological orders:] ~\\
\textnormal{%
The topological orders are:
\begin{itemize}
\item
$(\{r_1\},\{r_2\},\{r_3\},\{r_4\},\{r_5\})$,
\item
$(\{r_1\},\{r_2\},\{r_4\},\{r_5\},\{r_3\})$,
\item
$(\{r_1\},\{r_2\},\{r_4\},\{r_3\},\{r_5\})$,
\item
$(\{r_2\},\{r_1\},\{r_3\},\{r_4\},\{r_5\})$,
\item
$(\{r_2\},\{r_1\},\{r_4\},\{r_5\},\{r_3\})$, and
\item
$(\{r_2\},\{r_1\},\{r_4\},\{r_3\},\{r_5\})$.
\end{itemize}
In the exercise we only need to find one,
and in our case we choose \\ $L_{\Pi}=(\{r_1\},\{r_2\},\{r_3\},\{r_4\},\{r_5\})$.
}
%
\item[Sets $R_r$ for $r \in \Pi$:] ~\\
\textnormal{%
$R_{r_1}=\{{nselect(X)}\}$ and $R_{r_i}=\emptyset$ for $i \in \{2, \ldots, 5\}$.
}
%
\item[Grounding:] ~\
\textnormal{%
\begin{enumerate}
  \setcounter{enumi}{-1}
  \item Initialization: $F = I$, $D = I$
  \item Ground instantiation of component $\{r_1\}$:\footnote{%
  In this case, the grounder generates the same rule twice.\label{fnlabel}}
  \( \left\{
  \begin{array}{r@{}c@{}l}
    select(b) & {} \leftarrow {} & \naf{nselect(b)} \\
    select(c) & {} \leftarrow {} & \naf{nselect(c)} \\
    select(b) & {} \leftarrow {} & \naf{nselect(b)}
  \end{array}
  \right\}
  \)
  \\
  $F = I$ 
  \\ 
  $D = I \cup \{ select(b), select(c) \}$
  \item Ground instantiation of component $\{r_2\}$:\footref{fnlabel}
  \( \left\{
  \begin{array}{r@{}c@{}l}
    nselect(b) & {} \leftarrow {} & \naf{select(b)} \\
    nselect(c) & {} \leftarrow {} & \naf{select(c)} \\
    nselect(b) & {} \leftarrow {} & \naf{select(b)} \\
  \end{array}
  \right\}
  \)
  \\
  $F = I$
  \\
  $D = I \cup \{ select(b), select(c), nselect(b), nselect(c) \}$
  \item Ground instantiation of component $\{r_3\}$:\footref{fnlabel}
  \( \left\{
  \begin{array}{r@{}c@{}l}
    & {} \leftarrow {} & select(b), select(c) \\
    & {} \leftarrow {} & select(c), select(b)
  \end{array}
  \right\}
  \)
  \\
  $F = I$
  \\
  $D = I \cup \{ select(b), select(c), nselect(b), nselect(c) \}$
  \item Ground instantiation of component $\{r_4\}$:
  \( \left\{
  \begin{array}{r@{}c@{}l}
    hit(2) & {} \leftarrow {} & select(b)\\
    hit(2) & {} \leftarrow {} & select(c)\\
    hit(3) & {} \leftarrow {} & select(b)
  \end{array}
  \right\}
  \)
  \\
  $F = I$ 
  \\ 
  $D = I \cup \{ select(b), select(c), nselect(b), nselect(c), hit(2), hit(3) \}$
  \item Ground instantiation of component $\{r_5\}$:\footnote{%
  The grounder generates the empty integrity constraint, 
  and in this way it proves that the program is unsatisfiable.}
  \( \left\{
  \begin{array}{r@{}c@{}l}
    & {} \leftarrow {} & \\
    & {} \leftarrow {} & \naf{hit(2)}\\
    & {} \leftarrow {} & \naf{hit(3)}\\
  \end{array}
  \right\}
  \)
  \\
  $F = I$ 
  \\ 
  $D = I \cup \{ select(b), select(c), nselect(b), nselect(c), hit(2), hit(3) \}$
\end{enumerate}
}
%
\end{description}
%
% c
\item 
\begin{description}
%
\item[Dependency Graph:] ~\\
\(
  G_{\Pi} = \big( \Pi, \{(r_1,r_2), (r_1,r_3), (r_2,r_4), (r_3,r_4)\} \big)
\)
%
\item[Positive Dependency Graph:] ~\\
\(
  G_{\Pi}^+ = \big( \Pi, \{(r_1,r_2), (r_1,r_3)\} \big)
\)
%
\item[Topological orders:] ~\\
\textnormal{%
The topological orders are:
\begin{itemize}
\item
$(\{r_1\},\{r_2\},\{r_3\},\{r_4\})$, and
\item
$(\{r_1\},\{r_3\},\{r_2\},\{r_4\})$.
\end{itemize}
In the exercise we only need to find one,
and in our case we choose \\ $L_{\Pi}=(\{r_1\},\{r_2\},\{r_3\},\{r_4\})$.
}
%
\item[Sets $R_r$ for $r \in \Pi$:] ~\\
\textnormal{%
$R_{r_i}=\emptyset$ for $i \in \{1, \ldots, 4\}$.
}
%
\item[Grounding:] ~\
\textnormal{%
\begin{enumerate}
  \setcounter{enumi}{-1}
  \item Initialization: $F = I$, $D = I$
  \item Ground instantiation of component $\{r_1\}$:
  \( \left\{
  \begin{array}{r@{}c@{}l}
    \{ select(2) \} & {} \leftarrow {} & \\
    \{ select(3) \} & {} \leftarrow {} & \\
    \{ select(4) \} & {} \leftarrow {} & \\
  \end{array}
  \right\}
  \)
  \\
  $F = I$ 
  \\ 
  $D = I \cup \{ select(2..4) \}$
  \item Ground instantiation of component $\{r_2\}$:
  \( \left\{
  \begin{array}{r@{}c@{}l}
    covered(1,2) & {} \leftarrow {} & \\
    covered(2,3) & {} \leftarrow {} & select(2) \\
    covered(3,1) & {} \leftarrow {} & select(3) \\
  \end{array}
  \right\}
  \)
  \\
  $F = I \cup \{ covered(1,2) \}$
  \\
  $D = I \cup \{ select(2..4), covered(1,2), covered(2,3), covered(3,1) \}$
  \item Ground instantiation of component $\{r_3\}$:
  \( \left\{
  \begin{array}{r@{}c@{}l}
    covered(2,3) & {} \leftarrow {} & select(3) \\
    covered(3,1) & {} \leftarrow {} & \\
  \end{array}
  \right\}
  \)
  \\
  $F = I \cup \{ covered(1,2), covered(3,1) \}$
  \\
  $D = I \cup \{ select(2..4), covered(1,2), covered(2,3), covered(3,1) \}$
  \item Ground instantiation of component $\{r_4\}$:
  \( \left\{
  \begin{array}{r@{}c@{}l}
    & {} \leftarrow {} & \naf{covered(2,3)}\\
  \end{array}
  \right\}
  \)
  \\
  $F = I \cup \{ covered(1,2), covered(3,1) \}$
  \\
  $D = I \cup \{ select(2..4), covered(1,2), covered(2,3), covered(3,1) \}$
\end{enumerate}
}
%
\end{description}
%
% d
\item 
\begin{description}
%
\item[Dependency Graph:] ~\\
\(
  G_{\Pi} = \big( \Pi, \{(r_1,r_4), (r_2,r_4), (r_3,r_4), (r_4,r_5)\} \big)
\)
%
\item[Positive Dependency Graph:] ~\\
\(
  G_{\Pi}^+ = \big( \Pi, \{\} \big)
\)
%
\item[Topological orders:] ~\\
\textnormal{%
The topological orders are:
\begin{itemize}
\item
$(\{r_1\},\{r_2\},\{r_3\},\{r_4\},\{r_5\})$,
\item
$(\{r_1\},\{r_3\},\{r_2\},\{r_4\},\{r_5\})$,
\item
$(\{r_2\},\{r_1\},\{r_3\},\{r_4\},\{r_5\})$,
\item
$(\{r_2\},\{r_3\},\{r_1\},\{r_4\},\{r_5\})$,
\item
$(\{r_3\},\{r_1\},\{r_2\},\{r_4\},\{r_5\})$, and
\item
$(\{r_3\},\{r_2\},\{r_1\},\{r_4\},\{r_5\})$.
\end{itemize}
In the exercise we only need to find one,
and in our case we choose \\ $L_{\Pi}=(\{r_1\},\{r_2\},\{r_3\},\{r_4\},\{r_5\})$.
}
%
\item[Sets $R_r$ for $r \in \Pi$:] ~\\
\textnormal{%
$R_{r_i}=\emptyset$ for $i \in \{1, \ldots, 5\}$.
}
%
\item[Grounding:] ~\
\textnormal{%
\begin{enumerate}
  \setcounter{enumi}{-1}
  \item Initialization: $F = I$, $D = I$
  \item Ground instantiation of component $\{r_1\}$:
  \( \left\{
  \begin{array}{r@{}c@{}l}
    dominated(1) & {} \leftarrow {} & \\
    dominated(4) & {} \leftarrow {} & \\
  \end{array}
  \right\}
  \)
  \\
  $F = I \cup \{dominated(1), dominated(4) \}$
  \\ 
  $D = I \cup \{dominated(1), dominated(4) \}$
  \item Ground instantiation of component $\{r_2\}$:
  \( \left\{
  \begin{array}{r@{}c@{}l}
    dominated(3) & {} \leftarrow {} & \\
  \end{array}
  \right\}
  \)
  \\
  $F = I \cup \{dominated(1), dominated(3..4) \}$
  \\ 
  $D = I \cup \{dominated(1), dominated(3..4) \}$
  \item Ground instantiation of component $\{r_3\}$:
  \( \left\{
  \begin{array}{r@{}c@{}l}
    dominated(2) & {} \leftarrow {} & \\
  \end{array}
  \right\}
  \)
  \\
  $F = I \cup \{dominated(1..4) \}$
  \\
  $D = I \cup \{dominated(1..4) \}$
  \item Ground instantiation of component $\{r_4\}$:
  \( \left\{
  \right\}
  \)
  \\
  $F = I \cup \{dominated(1..4) \}$
  \\
  $D = I \cup \{dominated(1..4) \}$
  \item Ground instantiation of component $\{r_5\}$:
  \( \left\{
  \begin{array}{r@{}c@{}l}
    correct & {} \leftarrow {} & \\
  \end{array}
  \right\}
  \)
  \\
  $F = I \cup \{dominated(1..4), correct \}$
  \\
  $D = I \cup \{dominated(1..4), correct \}$
\end{enumerate}
}
%
\end{description}
%
% e
\item 
\begin{description}
%
\item[Dependency Graph:] ~\\
\(
  G_{\Pi} = \big( \Pi, \{(r_1,r_1), (r_1,r_2), (r_2,r_3)\} \big)
\)
%
\item[Positive Dependency Graph:] ~\\
\(
  G_{\Pi}^+ = \big( \Pi, \{(r_1,r_2)\} \big)
\)
%
\item[Topological orders:] ~\\
\textnormal{%
The unique topological order is $L_{\Pi}=(\{r_1\},\{r_2\},\{r_3\})$.
}
%
\item[Sets $R_r$ for $r \in \Pi$:] ~\\
\textnormal{%
$R_{r_1}=\{{holds(V2,A)}\}$ and $R_{r_i}=\emptyset$ for $i \in \{2,3\}$.
}
%
\item[Grounding:] ~\
\textnormal{%
\begin{enumerate}
  \setcounter{enumi}{-1}
  \item Initialization: $F = I$, $D = I$
  \item Ground instantiation of component $\{r_1\}$:
  \( \left\{
  \begin{array}{r@{}c@{}l}
  holds(neg,b) & {} \leftarrow {} & \naf{holds(pos,b)} \\
  holds(neg,c) & {} \leftarrow {} & \naf{holds(pos,c)} \\
  holds(pos,b) & {} \leftarrow {} & \naf{holds(neg,b)} \\
  holds(pos,c) & {} \leftarrow {} & \naf{holds(neg,c)}
  \end{array}
  \right\}
  \)
  \\
  $F = I$ 
  \\ 
  $D = I \cup \{ holds(neg,b), holds(neg,c), holds(pos,b), holds(pos,c) \}$
  \item Ground instantiation of component $\{r_2\}$:
  \( \left\{
  \begin{array}{r@{}c@{}l}
    satisfied(1) & {} \leftarrow {} & \\
    satisfied(2) & {} \leftarrow {} & {holds(pos,b)}\\
    satisfied(3) & {} \leftarrow {} & {holds(neg,b)}\\
    satisfied(3) & {} \leftarrow {} & {holds(pos,c)}\\
  \end{array}
  \right\}
  \)
  \\
  $F = I \cup \{ satisfied(1) \}$ 
  \\ 
  $D = I \cup \{ holds(neg,b), holds(neg,c), holds(pos,b), holds(pos,c), satisfied(1..3) \}$
  \item Ground instantiation of component $\{r_3\}$:
  \( \left\{
  \begin{array}{r@{}c@{}l}
    & {} \leftarrow {} & \naf{satisfied(2)} \\
    & {} \leftarrow {} & \naf{satisfied(3)}
  \end{array}
  \right\}
  \)
  \\
  $F = I \cup \{ satisfied(1) \}$ 
  \\ 
  $D = I \cup \{ holds(neg,b), holds(neg,c), holds(pos,b), holds(pos,c), satisfied(1..3) \}$
\end{enumerate}
}
%
\end{description}
\end{UList}
%}
\end{Loesung}

%%% Local Variables: 
%%% mode: latex
%%% TeX-master: "uebung04"
%%% End: 
