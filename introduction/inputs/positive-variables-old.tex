\begin{Uebung}{(Positive Logic Programs with Variables)}%
Determine the Herbrand models of the following positive logic programs with variables
and decide which Herbrand models are stable.

\begin{UList}
\item
\(
P =
\left\{
\begin{array}{r@{{}\leftarrow{}}l}
    \mathit{fish}(\mathit{blinky})        \\
    \mathit{bird}(\mathit{tweety})      \\
%     \mathit{bird}(X)  & \mathit{sparrow}(X) \\
%     \mathit{bird}(X)  & \mathit{penguin}(X) \\
%     \mathit{bird}(X)  & \mathit{vulture}(X) \\
    \mathit{flies}(X) & \mathit{bird}(X) % , \naf{\mathit{penguin}(X)}
\end{array}
\right\}
\)
\item 
\(
P =
\left\{
\begin{array}{r@{{}\leftarrow{}}l}
    \mathit{next}(0,1) \\
    \mathit{next}(1,0) \\
    \mathit{even}(0)  \\
    \mathit{even}(Y) & \mathit{next}(X,Y), \mathit{odd}(X)  \\
    \mathit{odd}(Y)  & \mathit{next}(X,Y), \mathit{even}(X)
\end{array}
\right\}
\)
\item
\(
P =
\left\{
\begin{array}{r@{{}\leftarrow{}}l}
    \mathit{friend}(\mathit{alice},\mathit{bob})  \\
    \mathit{friend}(\mathit{bob},\mathit{alice})  \\
    \mathit{friend}(\mathit{eve},\mathit{alice})  \\
    \mathit{invite}(\mathit{alice})  \\
%     \mathit{invite}(X) & \naf{\mathit{refuse}(X)} \\
%     \mathit{refuse}(X) & \naf{\mathit{invite}(X)} \\
    \mathit{invite}(Y) & \mathit{invite}(X), \mathit{friend}(X,Y)
\end{array}
\right\}
\)
\item
\(
P =
\left\{
\begin{array}{r@{{}\leftarrow{}}l}
    \mathit{next}(0,1) \\
    \mathit{next}(1,2) \\
    \mathit{before}(X)   & \mathit{next}(X,Y) \\
    \mathit{between}(Y)  & \mathit{next}(X,Y), \mathit{before}(Y)
\end{array}
\right\}
\)
\end{UList}
\end{Uebung}

\begin{Loesung}\textnormal{%
\begin{UList}
\item
Herbrand Models:
\begin{itemize}
\item $\{\mathit{fish}(\mathit{blinky}),\mathit{bird}(\mathit{tweety}),\mathit{flies}(\mathit{tweety})\}$
\item $\{\mathit{fish}(\mathit{blinky}),\mathit{bird}(\mathit{tweety}),\mathit{flies}(\mathit{tweety}),\mathit{flies}(\mathit{blinky})\}$
\item $\{\mathit{fish}(\mathit{blinky}),\mathit{bird}(\mathit{tweety}),\mathit{flies}(\mathit{tweety}),\mathit{fish}(\mathit{tweety})\}$
\item $\{\mathit{fish}(\mathit{blinky}),\mathit{bird}(\mathit{tweety}),\mathit{flies}(\mathit{tweety}),\mathit{fish}(\mathit{tweety}),\mathit{flies}(\mathit{blinky})\}$
\item $\{\mathit{fish}(\mathit{blinky}),\mathit{bird}(\mathit{tweety}),\mathit{flies}(\mathit{tweety}),\mathit{bird}(\mathit{blinky}),\mathit{flies}(\mathit{blinky})\}$
\item $\{\mathit{fish}(\mathit{blinky}),\mathit{bird}(\mathit{tweety}),\mathit{flies}(\mathit{tweety}),\mathit{bird}(\mathit{blinky}),\mathit{flies}(\mathit{blinky}),\mathit{fish}(\mathit{tweety})\}$
\end{itemize}
Stable Herbrand Model: $\{\mathit{fish}(\mathit{blinky}),\mathit{bird}(\mathit{tweety}),\mathit{flies}(\mathit{tweety})\}$
%
\item
Herbrand Models:
\begin{itemize}
\item $\{\mathit{next}(0,1),\mathit{next}(1,0),\mathit{even}(0),\mathit{odd}(1)\}$
\item $\{\mathit{next}(0,1),\mathit{next}(1,0),\mathit{even}(0),\mathit{odd}(1),\mathit{even}(1),\mathit{odd}(0)\}$
\item $\{\mathit{next}(0,1),\mathit{next}(1,0),\mathit{even}(0),\mathit{odd}(1),\mathit{even}(1),\mathit{odd}(0),\mathit{next}(0,0)\}$
\item $\{\mathit{next}(0,1),\mathit{next}(1,0),\mathit{even}(0),\mathit{odd}(1),\mathit{even}(1),\mathit{odd}(0),\mathit{next}(1,1)\}$
\item $\{\mathit{next}(0,1),\mathit{next}(1,0),\mathit{even}(0),\mathit{odd}(1),\mathit{even}(1),\mathit{odd}(0),\mathit{next}(0,0),\mathit{next}(1,1)\}$
\end{itemize}
Stable Herbrand Model: $\{\mathit{next}(0,1),\mathit{next}(1,0),\mathit{even}(0),\mathit{odd}(1)\}$
%
\item
Stable Herbrand Model: \[X = \left\{\begin{array}{@{}l@{}}\mathit{friend}(\mathit{alice},\mathit{bob}),\mathit{friend}(\mathit{bob},\mathit{alice}),\mathit{friend}(\mathit{eve},\mathit{alice}),{}\\\mathit{invite}(\mathit{alice}),\mathit{invite}(\mathit{bob})\end{array}\right\}\]
We can construct 80 different Herbrand models $Y\supseteq X$ of~$P$ by
extending the stable Herbrand model~$X$ as described in the following.
There are 16 subsets
\[Z\subseteq\{\mathit{friend}(\mathit{alice},\mathit{alice}),\mathit{friend}(\mathit{bob},\mathit{bob}),
              \mathit{friend}(\mathit{eve},\mathit{eve}),\mathit{friend}(\mathit{eve},\mathit{bob})\}\]
for which
$Y=X\cup Z$,
$Y=X\cup Z\cup\{\mathit{invite}(\mathit{eve})\}$,
$Y=X\cup Z\cup\{\mathit{invite}(\mathit{eve}),\linebreak[1]\mathit{friend}(\mathit{alice},\mathit{eve})\}$,
$Y=X\cup Z\cup\{\mathit{invite}(\mathit{eve}),\linebreak[1]\mathit{friend}(\mathit{bob},\mathit{eve})\}$, and
$Y=X\cup Z\cup\{\mathit{invite}(\mathit{eve}),\linebreak[1]\mathit{friend}(\mathit{alice},\mathit{eve}),\linebreak[1]\mathit{friend}(\mathit{bob},\mathit{eve})\}$
are Herbrand models of~$P$, which amount to 80 Herbrand models $Y\supseteq X$ in total.
%
\item
Stable Herbrand Model: \[X = \left\{\mathit{next}(0,1),\mathit{next}(1,2),\mathit{before}(0),\mathit{before}(1),\mathit{between}(1)\right\}\]
We can construct 184 different Herbrand models $Y\supseteq X$ of~$P$ by
extending the stable Herbrand model~$X$ as described in the following.
There are 4 subsets
\[Z\subseteq\{\mathit{next}(0,2),\mathit{next}(1,1)\}\]
for which
\begin{enumerate}
\item $Y_1=X\cup Z$,
\item $Y_2=X\cup Z\cup\{\mathit{between}(2)\}$,
\item $Y_3=X\cup Z\cup\{\mathit{between}(0)\}$,
\item $Y_4=X\cup Z\cup\{\mathit{between}(2),\mathit{between}(0)\}$,
\item $Y_5=X\cup Z\cup\{\mathit{between}(2),\mathit{before}(2)\}$, and
\item $Y_6=X\cup Z\cup\{\mathit{between}(2),\mathit{before}(2),\mathit{between}(0)\}$
\end{enumerate}
are Herbrand models of~$P$.
Each of these Herbrand models~$Y_i$ can be joined with any subset~$A$ of a corresponding set~$A_i$ of atoms
such that $Y=Y_i\cup A$ is a Herbrand model of~$P$, where $A_i$ is as follows for $1\leq i\leq 6$:
\begin{enumerate}
\item $A_1=\emptyset$,
\item $A_2=\emptyset$,
\item $A_3=\{\mathit{next}(0,0),\mathit{next}(1,0)\}$,
\item $A_4=\{\mathit{next}(0,0),\mathit{next}(1,0)\}$,
\item $A_5=\{\mathit{next}(2,1),\mathit{next}(2,2)\}$, and
\item $A_6=\{\mathit{next}(0,0),\mathit{next}(1,0),\mathit{next}(2,0),\mathit{next}(2,1),\mathit{next}(2,2)\}$.
\end{enumerate}
Now the total number of Herbrand models $Y\supseteq X$ can be calculated as follows:
\[
\begin{array}{r@{}c@{}l}
 4*(2^{|A_1|}+2^{|A_2|}+2^{|A_3|}+2^{|A_4|}+2^{|A_5|}+2^{|A_6|}) & {} = {} \\
 4*(2^0+2^0+2^2+2^2+2^2+2^5) & {} = {} \\
 4*(1+1+4+4+4+32) & {} = {} \\
 4*46 & {} = {} & 184
 \mbox{.}
\end{array}
\]
\end{UList}}
\end{Loesung}

%%% Local Variables: 
%%% mode: latex
%%% TeX-master: "exercise01"
%%% End: 
