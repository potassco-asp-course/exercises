\begin{Uebung}{(Basic language constructs)}%
%
Consider the following logic program $P$:
\begin{itemize}
\item[]
\(
P =
\left\{
\begin{array}{r@{{}\leftarrow{}}l}
\{ \mathit{a} \} &
\\
\{ \mathit{b} \} & a
\\
\mathit{d}      & 2 \{ \mathit{a}, \mathit{b}, \neg\mathit{c} \}
\\
& \neg\mathit{d}
\end{array}
\right\}
\)
\end{itemize}
\begin{UList}
\item
Compile~$P$ into a normal logic program~$P'$ 
using the translations from the lecture slides.
%
In particular, use the
$\mathit{x}(i,j)$ construction to translate the cardinality rule
of the program.
\item
Determine the stable models of~$P$ and the corresponding stable models of~$P'$.
\end{UList}
\end{Uebung}

%a^*
%a :-  a^*, \neg a'.
%a' :-    \neg a.
%b' :- a.
%b :-  b', \neg b''.
%b'' :-    \neg b.
%d :- x(1,2).
%d' :- not d', not d.x


\begin{Loesung}%
\rm
\begin{UList}
\item $P':{}$
\begin{equation*}
\begin{array}[t]{r@{{}\leftarrow{}}l@{\qquad}r@{{}\leftarrow{}}l@{\quad}r@{{}\leftarrow{}}l@{\quad}r@{{}\leftarrow{}}l}
%\multicolumn{2}{l}
a^* &
&
\mathit{x}(1,0) & \mathit{x}(2,0).
&
\mathit{x}(2,0) & \mathit{x}(3,0).
&
\mathit{x}(3,0) & \mathit{x}(4,0).
\\
a & a^*, \neg a'. %& \mathit{x}(1,1).
&
\mathit{x}(1,1) & \mathit{x}(2,0), a.
&
\mathit{x}(2,1) & \mathit{x}(3,0), b.
&
\mathit{x}(3,1) & \mathit{x}(4,0), \neg c.
\\
a' & \neg{a}.
&
\mathit{x}(1,1) & \mathit{x}(2,1).
&
\mathit{x}(2,1) & \mathit{x}(3,1).
&
[\mathit{x}(3,1) & \mathit{x}(4,1).]
\\
b^* & a.
&
\mathit{x}(1,2) & \mathit{x}(2,1), a.
&
\mathit{x}(2,2) & \mathit{x}(3,1), b.
&
[\mathit{x}(3,2) & \mathit{x}(4,1), \neg c.]
\\
b & b^*, \neg b'.
&
\mathit{x}(1,2) & \mathit{x}(2,2).
&
[\mathit{x}(2,2) & \mathit{x}(3,2).]
&

[\mathit{x}(3,2) & \mathit{x}(4,2).]
\\
b' &    \neg b. 
%\multicolumn{5}{@{}l}{\phantom{[}\mathit{x}(4,0).}
& \multicolumn{5}{r@{{}\leftarrow{}}}{x(4,0)} & .%{\mathit{x}(1,3)} & \mathit{x}(2,2),a.
%& 
%[\mathit{x}(2,3) & \mathit{x}(3,2),b.]
\\
d & x(1,2). 
\\
d' & \neg d', \neg d.
%\multicolumn{3}{r@{{}\leftarrow{}}}{
%[\mathit{x}(1,3) & \mathit{x}(2,3).]
%&
%[\mathit{x}(2,3) & \mathit{x}(3,3).]
%&
%\multicolumn{2}{@{}l}{\phantom{[}\mathit{x}(4,0).}
\end{array}
\end{equation*}

%b'' :-    \neg b.
%d :- x(1,2).
%d' :- not d', not d.x

\item
\begin{equation*}
\begin{array}{l|l}
\multicolumn{1}{c|}{P} & \multicolumn{1}{c}{P'}
\\\hline
\{a,d\}
&
\{a,d,a^*,b^*,b',\mathit{x}(1,0),\mathit{x}(2,0),\mathit{x}(3,0),\mathit{x}(4,0),\mathit{x}(1,1),\mathit{x}(2,1),
\mathit{x}(3,1),\mathit{x}(1,2)\}
\\
\{a,b,d\}
&
\{a,b,d,a^*,b^*,\mathit{x}(1,0),\mathit{x}(2,0),\mathit{x}(3,0),\mathit{x}(4,0),\mathit{x}(1,1),\mathit{x}(2,1),
\mathit{x}(3,1),\mathit{x}(1,2),\mathit{x}(2,2)\}
\end{array}
\end{equation*}
\end{UList}
\newpage
\end{Loesung}

%%% Local Variables: 
%%% mode: latex
%%% TeX-master: "exercise03"
%%% End: 
