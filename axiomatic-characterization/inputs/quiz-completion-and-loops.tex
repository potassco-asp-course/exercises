%%%%%%%%%%%%%%%%%%%%%%%%%%%%%%%%%%%%%%%%%%%%%%%%%%%%%%%%%%%%%%%%%%%%%%%%%%%
% Quizaufgaben

\begin{quiz}
\quizAufgabe{%Sei~$\Pi$ ein normales Programm und~$X$ eine Menge von Atomen.
             If a set of atoms $X$ is a model of a normal program~$\Pi$, 
             then $X$ is a model of $\Comp{\Pi}$.} %\footnotemark}
\quizAufgabe{%Sei~$\Pi$ ein normales Programm und~$X$ eine Menge von Atomen.
             Given a normal program~$\Pi$, 
             if a set of atoms $X$ is a model of $\Comp{\Pi}$,
             then $X$ is a model of~$\Pi$.}
\quizAufgabe{Given a normal program~$\Pi$, if a set of atoms $X$ is a model of~$\Comp{\Pi}$, 
             then $X$ is a stable model of~$\Pi$.}
\quizAufgabe{%Sei~$\Pi$ ein normales Programm und~$X$ eine Menge von Atomen.
             If $X$ is a stable model of a normal program~$\Pi$,
             then $X$ is a model of~$\Comp{\Pi}$.}
% \quizAufgabe{Let~$\Pi$ ein normales Programm und~$X$ ein Modell von $\Comp{\Pi}$.
%              F\"ur alle Mengen $T\subseteq X,F\subseteq (\atom{\Pi}\setminus X)$
%              gilt 
%              $T'\subseteq X,F'\subseteq (\atom{\Pi}\setminus\nolinebreak X)$,
%              wenn $\langle T',F' \rangle=\FiPi{\Pi}{\langle T,F \rangle}$.}
% \quizAufgabe{Sei~$\Pi$ ein normales Programm.
%              F\"ur alle Mengen $T,F$ von Atomen,
%              sodass $T\subseteq T',F\subseteq F'$, wenn
%              $\langle T',F' \rangle=\FiPi{\Pi}{\langle T,F \rangle}$,
%              gibt es ein Modell~$X$ von $\Comp{\Pi}$ mit
%              $T'\subseteq X,F'\subseteq (\atom{\Pi}\setminus X)$.}
\quizAufgabe{% Sei~$\Pi$ ein normales Programm.
             Given a normal program~$\Pi$, 
             if~$\Comp{\Pi}$ has (at least) one model,
             then $\Pi$ has (at least) one stable model.}
%%%\quizAufgabe{The decision problem, if for a normal program~$\Pi$
%%%             there is (at least) one model of $\Comp{\Pi}$,
%%%             is NP-complete.}

% \quizAufgabe{Sei~$\Pi$ ein normales Programm und~$\langle T,F \rangle\sqsubseteq\langle T',F' \rangle$
%              partielle Interpretationen.
%              Wenn eine Menge $U\subseteq\atom{\Pi}$ f�r $\langle T,F \rangle$ unfundiert ist,
%              dann ist $U$ auch f�r $\langle T',F' \rangle$ unfundiert.}
% \quizAufgabe{Sei~$\Pi$ ein normales Programm und~$\langle T,F \rangle$ eine
%              partielle Interpretation.
%              Wenn $U_1,U_2\subseteq\atom{\Pi}$ f�r $\langle T,F \rangle$ unfundiert sind,
%              dann ist auch $(U_1\cap U_2)$ f\"ur $\langle T,F \rangle$ unfundiert.}
\quizAufgabe{Let~$\Pi$ be a normal program and~$X$ a set of atoms such that~$X\models\Comp{\Pi}$.
             If $X$ is not a stable model of~$\Pi$,
             then there is a loop $L\in\Loop{\Pi}$
             such that $L\subseteq X$ and $X\not\models\LFP{\Pi}{L}$.}
\quizAufgabe{Let~$\Pi$ be a normal program and $L_1,L_2\in\Loop{\Pi}$.
             If $(L_1\cap\nolinebreak L_2)\neq\nolinebreak\emptyset$,
             then $(L_1\cup L_2)\in\Loop{\Pi}$.}
\end{quiz}
%\footnotetext{%
%Given a normal program $\Pi$, 
%the Clark's Completion $\Comp{\Pi}$ 
%is the following set of propositional formulas: \
%$\Comp{\Pi}=\big\{\;a\leftrightarrow\big(
%    \mbox{$\bigvee$}_{r\in\Pi, \head{r} = a}(
%    \mbox{$\bigwedge$}_{p \in \pbody{r}} p \wedge
%    \mbox{$\bigwedge$}_{q \in \nbody{r}} \neg q)\big)\;\mid\; a\in\atom{\Pi}\;\big\}$.}

%%%%%%%%%%%%%%%%%%%%%%%%%%%%%%%%%%%%%%%%%%%%%%%%%%%%%%%%%%%%%%%%%%%%%%%%%%%
% L�sungen

\begin{Loesung}
\begin{quiz}
\quizAufgabe[false]{We give a counterexample:
                     The set $X=\{a\}$ is a model of
                     $\Pi=\{a\leftarrow\naf{a}\}$, but not of
                     $\Comp{\Pi}=\{a\leftrightarrow\neg a\}$.}
\quizAufgabe[true]{%
  If~$X$ is a model of $\Comp{\Pi}$,
  then for every atom $a\in\atom{\Pi}$ it holds that 
  $$X\models a\leftrightarrow\big(
    \mbox{$\bigvee$}_{r\in\Pi, \head{r} = a}(
    \mbox{$\bigwedge$}_{p \in \pbody{r}} p \wedge
    \mbox{$\bigwedge$}_{q \in \nbody{r}} \neg q)\big)$$
  This implies that 
  $$X\models (
    \mbox{$\bigwedge$}_{p \in \pbody{r}} p \wedge
    \mbox{$\bigwedge$}_{q \in \nbody{r}} \neg q)
    \rightarrow \head{r}$$
  for every rule $r\in\Pi$, so $X$ is a model of $P$.}
\quizAufgabe[false]{%
  We give a counterexample: 
  Given the program $\Pi=\{a\leftarrow a\}$,
  $X=\{a\}$ is a model of $\Comp{\Pi}=\{a\leftrightarrow a\}$,
  while $X$ is not a stable model of~$\Pi$.}
\quizAufgabe[true]{%
  If~$X$ is a stable model of $\Pi$,
  then $X$ is a model of $\Pi$, therefore\  
  $$X\models\big(
    \mbox{$\bigvee$}_{r\in\Pi, \head{r} = a}(
    \mbox{$\bigwedge$}_{p \in \pbody{r}} p \wedge
    \mbox{$\bigwedge$}_{q \in \nbody{r}} \neg q)\big)
    \rightarrow a$$
  holds for every atom $a\in\atom{\Pi}$.
  Moreover, for every atom $a\in\nolinebreak X$ there is a ``supporting'' rule
  $r\in\Pi$ with $\head{r}=a$, $\pbody{r}\subseteq\nolinebreak X$ and $\nbody{r}\cap X=\emptyset$, therefore\
  $$X\models
    a \rightarrow \big(
    \mbox{$\bigvee$}_{r\in\Pi, \head{r} = a}(
    \mbox{$\bigwedge$}_{p \in \pbody{r}} p \wedge
    \mbox{$\bigwedge$}_{q \in \nbody{r}} \neg q)
    \big)$$
  holds for every atom $a\in\atom{\Pi}$, so $X$ is a model of $\Comp{\Pi}$.}
%  sodass 
%   $$X\models\big(x\leftrightarrow
%     \bigvee_{r\in\Pi, \head{r} = x}(
%     \mbox{$\bigwedge$}_{p \in \pbody{r}} p \wedge
%     \mbox{$\bigwedge$}_{q \in \nbody{r}} \neg q)\big)\ \text{.}$$
%   Wir unterscheiden die folgenden beiden F\"alle:
%   \begin{enumerate}
%   \item Wenn $x\notin X$, dann gibt es (mindestens) eine Regel $r\in\Pi$,
%         sodass $\head{r}=x$, $\pbody{r}\subseteq X$ und $\nbody{r}\cap X=\emptyset$.
%         F\"ur jede dieser Regeln $r\in\Pi$ gilt, dass
%         $(x\leftarrow\pbody{r})\in\reduct{\Pi}{X}$,
%         $\pbody{r}\subseteq X$ und $(\head{r}=x)\notin X$.
%         Daraus schlie{\ss}en wir, dass $X$ nicht unter $\reduct{\Pi}{X}$ abgeschlossen ist,
%         sodass $X\neq\Cn{\reduct{\Pi}{X}}$.
%   \item Wenn $x\in X$, dann gilt
%         $\head{r}\neq x$,
%         $\pbody{r}\not\subseteq X$ oder $\nbody{r}\cap X\neq\emptyset$
%         f\"ur jede Regel $r\in\Pi$.
%         Das bedeutet, dass $\head{r}\neq x$ oder $\pbody{r}\not\subseteq X$
%         f\"ur jede Regel $r\in\reduct{\Pi}{X}$ gilt.
%         Wenn $X$ unter $\reduct{\Pi}{X}$ abgeschlossen ist,
% %        (d.h.\ $\pbody{r}\subseteq X$ impliziert $\head{r}\in X$ f\"ur alle $r\in\reduct{\Pi}{X}$),
%         dann ist auch $X\setminus\{x\}$ unter $\reduct{\Pi}{X}$ abgeschlossen,
%         da $\pbody{r}\subseteq(X\setminus\{x\}) \subseteq X$ und $\head{r}\neq x$
%         in diesem Fall $\head{r}\in (X\setminus\{x\})$ implizieren.
%         Daraus schlie{\ss}en wir, dass $X\neq\Cn{\reduct{\Pi}{X}}$.
%   \end{enumerate}
%   Da in beiden F\"allen $X\neq\Cn{\reduct{\Pi}{X}}$ gilt,
%   ist $X$ keine Antwortmenge von~$\Pi$.
% }
\quizAufgabe[false]{%
  We give a counterexample: 
  For the program $\Pi=\{a\leftarrow\nolinebreak a\quad\linebreak[1] a\leftarrow\nolinebreak\naf{a}\}$,
  $X=\{a\}$ is a model of $\Comp{\Pi}=\{a\leftrightarrow (a\vee\neg a)\}$,
  but $\Pi$ has no stable models.}
% \quizAufgabe[richtig]{%
%   F\"ur jedes Atom $x\in T'$ gibt es (mindestens) eine Regel $r\in\Pi$,
%   sodass $x=\head{r}$, $\pbody{r}\subseteq T\subseteq X$ und
%   $\nbody{r}\subseteq F\subseteq (\atom{\Pi}\setminus X)$,
%   d.h.
%   $X\models (
%     \mbox{$\bigwedge$}_{p \in \pbody{r}} p \wedge
%     \mbox{$\bigwedge$}_{q \in \nbody{r}} \neg q)
%   $
%   gilt.
%   Da $X$ ein Modell von $\Comp{\Pi}$ ist,
%   folgt daraus $x\in X$.\newline
%   F\"ur jedes Atom $x\in F'$ und jede Regel $r\in\Pi$,
%   sodass $x=\head{r}$, 
%   gilt $\pbody{r}\cap F\subseteq\pbody{r}\cap (\atom{\Pi}\setminus X)\neq\emptyset$ oder
%   $\nbody{r}\cap T\subseteq\nbody{r}\cap X\neq\emptyset$,
%   d.h.
%   $X\not\models (
%     \mbox{$\bigwedge$}_{p \in \pbody{r}} p \wedge
%     \mbox{$\bigwedge$}_{q \in \nbody{r}} \neg q)
%   $
%   gilt.
%   Da $X$ ein Modell von $\Comp{\Pi}$ ist,
%   folgt daraus $x\notin X$.
% }
% \quizAufgabe[falsch]{%
%   Gegenbeispiel: F\"ur $\Pi=\{x\leftarrow\mathit{not}~x\}$
%   gilt $\FiPi{\Pi}{\langle \emptyset,\emptyset \rangle}=\langle \emptyset,\emptyset \rangle$,
%   aber es gibt kein Modell~$X$ von 
%   $\Comp{\Pi}=\{x\leftrightarrow \neg x\}$
%   (mit $\emptyset\subseteq X,\emptyset\subseteq (\atom{\Pi}\setminus X)=(\{x\}\setminus X)$).
% }
%%%\quizAufgabe[true]{%
%%%  Die Grundinstanzen 
%%%  des in den Vorlesungsfolien gegebenen Encodings der
%%%  Drei-F\"arbung von Graphen sind ``tight''.
%%%  F\"ur eine Grundinstanz~$\Pi$ korrespondieren die
%%%  Antwortmengen von~$\Pi$ daher zu den Modellen von $\Comp{\Pi}$,
%%%  und das Entscheidungsproblem, ob ein Graph eine Drei-F\"arbung hat,
%%%  ist bekannterma{\ss}en NP-vollst\"andig.
%%%  Das in den Vorlesungsfolien angegebene Encoding
%%%  (zusammen mit der polynomiellen \"Ubersetzbarkeit
%%%   von logischen Programmen mit Cardinality Constraints in normale Programme) stellt also
%%%  eine Reduktion des NP-vollst\"andigen Problems der Drei-F\"arbbarkeit von Graphen
%%%  auf die Frage dar, ob es f\"ur ein normales Programm~$\Pi$ (mindestens)
%%%             ein Modell von $\Comp{\Pi}$ gibt.
%%%  Weiterhin ist die Zugeh\"origkeit dieses Entscheidungsproblems zu der Klasse NP
%%%  ``offensichtlich'', da es einfach (d.h.\ mit polynomiellem Zeitaufwand m\"oglich) ist,
%%%  ein Modell von $\Comp{\Pi}$ zu verifizieren.
%%%% auf einfache Weise verifizierbar ist.
%%%}
% \quizAufgabe[richtig]{Falls eine Regel $r\in\Pi$ Bedingung 1\footnotemark[1]\ f\"ur $\langle T,F \rangle$
%                       erf\"ullt, dann erf\"ullt~$r$ Bedingung 1 auch bez\"uglich $\langle T',F' \rangle$,
%                       da $T\subseteq T'$ und $F\subseteq F'$.
%                       Ob Bedingung 2\footnotemark[2]\ gilt bzw.\ nicht gilt, h\"angt nicht von
%                       $\langle T,F \rangle$ bzw.\ $\langle T',F' \rangle$ ab.
%                       Falls $U$ f\"ur $\langle T,F \rangle$ unfundiert ist,
%                       d.h.\ Bedingung 1 oder 2 gilt f\"ur jede Regel $r\in\Pi$
%                       mit $\head{r}\in U$, dann gilt dieselbe Bedingung
%                       auch bez\"uglich $\langle T',F' \rangle$.
%                       Folglich ist $U$ auch f\"ur $\langle T',F' \rangle$ unfundiert.}
% \quizAufgabe[falsch]{Wir geben ein Gegenbeispiel an.
%                      Sei $\Pi=\{p\leftarrow q\quad q\leftarrow p,r\quad r\leftarrow q\}$.
%                      Die Mengen $U_1=\{p,q\}$ und $U_2=\{q,r\}$ sind f\"ur
%                      $\langle\emptyset,\emptyset\rangle$ unfundiert,
%                      nicht aber $(U_1\cap U_2)=\{q\}$.
%                      (Allerdings ist die Vereinigung $(U_1\cup U_2)$ zweier unfundierter
%                       Mengen $U_1,U_2$ wiederum unfundiert.)}
\quizAufgabe[true]{If $X\models\Comp{\Pi}$ is not a stable model of~$\Pi$,
                      then it holds that $X\not\models\LF{\Pi}$,
                      and therefore there exists $L\in\Loop{\Pi}$ with $X\not\models\LFP{\Pi}{L}$.
                      Now, from $X\not\models\LFP{\Pi}{L}$ we deduce that
                      $L\cap X\neq\emptyset$ and
                      $\pbody{r}\not\subseteq X$ or $\nbody{r}\cap X\neq\emptyset$ hold
                      for every rule
                      $r\in\ESP{\Pi}{L}=\{r\in\Pi\mid\head{r}\in L,\pbody{r}\cap L=\emptyset\}$.
                      Moreover, it holds that $\pbody{r}\cap(L\setminus X)\neq\emptyset$ 
                      and then $\pbody{r}\not\subseteq X$ holds for every rule
                      $r\in(\ESP{\Pi}{L\cap X}\setminus\ESP{\Pi}{L})$,
                      from where we can deduce $X\not\models\LFP{\Pi}{L\cap X}$.
                      Finally, the fact that 
                      $X\models\Comp{\Pi}$
                      implies the existence of a maximal loop
                      $L'\in\Loop{\Pi}$ with $L'\subseteq(L\cap X)$, so that
                      no atom in $(L\cap X)\setminus L'$ has an edge to an atom 
                      in $L'$ in $\G{\Pi}$.
                      This is equivalent to 
                      $\ESP{\Pi}{L'}\subseteq\ESP{\Pi}{L\cap X}$,
                      and together with $X\not\models\LFP{\Pi}{L\cap X}$,
                      we obtain $X\not\models\LFP{\Pi}{L'}$.}
\quizAufgabe[true]{If $p\in(L_1\cap L_2)$, then $p$ has a path to all atoms 
                      in $(L_1\cup L_2)$ in the subgraph of $\G{\Pi}$ induced by $(L_1\cup L_2)$,
                      $\G{\Pi}|_{(L_1\cup L_2)}$.
                      Moreover all atoms in $(L_1\cup L_2)$ have a path to $p$
                      in $\G{\Pi}|_{(L_1\cup L_2)}$.
                      This implies that $\G{\Pi}|_{(L_1\cup L_2)}$ is strongly connected,
                      and therefore $(L_1\cup L_2)\in\Loop{\Pi}$ holds.}
\end{quiz}
% \footnotetext[1]{$\pbody{r}\cap F\neq\emptyset$ oder $\nbody{r}\cap T\neq\emptyset$}
% \footnotetext[2]{$\pbody{r}\cap U\neq\emptyset$}
\newpage%\vspace*{-16mm}%
\end{Loesung}

%%% Local Variables: 
%%% mode: latex
%%% TeX-master: "uebung05"
%%% End: 
