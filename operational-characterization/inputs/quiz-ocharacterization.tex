%%%%%%%%%%%%%%%%%%%%%%%%%%%%%%%%%%%%%%%%%%%%%%%%%%%%%%%%%%%%%%%%%%%%%%%%%%%
% Quizaufgaben

\begin{quiz}
\quizAufgabe{For every normal logic program program~$P$ and every partial interpretation~$\langle T,F \rangle$
             there is at least one unfounded set of~$P$ with respect to~$\langle T, F \rangle$.}
\quizAufgabe{Let~$P$ be a normal program and~$\langle T,F\rangle$ be
             a partial interpretation.
             If \nolinebreak a set $U\subseteq\atom{P}$ is unfounded for~$P$ wrt~$\langle T,F \rangle$,
             then $(U\setminus F)$ is also    unfounded for~$P$ wrt~$\langle T,F \rangle$.}
\quizAufgabe{Let~$P$ be a normal program and~$\langle T,F \rangle\sqsubseteq\langle T',F' \rangle$
             be a partial interpretations.
             If \nolinebreak a set $U\subseteq\atom{P}$ is unfounded for~$P$ wrt~$\langle T ,F  \rangle$,
             then $U$ is also                 unfounded for~$P$ wrt~$\langle T',F' \rangle$.}
\quizAufgabe{Let~$P$ be a normal program and~$\langle T,F\rangle$
             be a partial interpretation.
             If \nolinebreak two sets $U_1,U_2\subseteq\atom{P}$ are unfounded for~$P$ wrt~$\langle T,F \rangle$,
             then \mbox{$(U_1\cap U_2)$} is also               unfounded for~$P$ wrt~$\langle T,F \rangle$.}
\quizAufgabe{Let~$P$ be a normal program and~$\langle T,F\rangle$
             be a partial interpretation.
             If \nolinebreak two sets $U_1,U_2\subseteq\atom{P}$ are unfounded for~$P$ wrt~$\langle T,F \rangle$,
             then \mbox{$(U_1\cup U_2)$} is also               unfounded for~$P$ wrt~$\langle T,F \rangle$.}
\quizAufgabe{Let~$P$ be a normal program and $X$ be a set of atoms.
             If \nolinebreak $(\atom{P}\setminus X)$ is an unfounded set of~$P$ wrt~$\langle X,\emptyset \rangle$, 
             then $X$ is a model of $P$.}
%\quizAufgabe{Let~$P$ be a normal program and~$\langle T,F\rangle$
%             be a partial interpretation.
%             The set $\Cn{\{\head{r}\leftarrow\pbody{r}\mid
%                            r\in P,
%                            (\pbody{r}\cap F)\cup(\nbody{r}\cap T)=\emptyset\}}$
%             is unfounded for~$P$ wrt~$\langle T,F \rangle$.}
\quizAufgabe{Let~$P$ be a normal program.
             For every partial interpretation~$\langle T,F \rangle$, if
             $\WiPi{P}\langle T,F \rangle=\langle T',F' \rangle$ then
             $T'\cap F'=\emptyset$.}
%\quizAufgabe{Let~$P$ be a normal program and~$X$ be a model of $\Comp{\Pi}$.
%             For every sets $T\subseteq X,F\subseteq (\atom{\Pi}\setminus X)$,
%             if $\langle T',F' \rangle=\FiPi{\Pi}{\langle T,F \rangle}$,
%             then it holds that
%             $T'\subseteq X,F'\subseteq (\atom{\Pi}\setminus\nolinebreak X)$.}
\end{quiz}

%%%%%%%%%%%%%%%%%%%%%%%%%%%%%%%%%%%%%%%%%%%%%%%%%%%%%%%%%%%%%%%%%%%%%%%%%%%
% L�sungen

\begin{Loesung}
\begin{quizLoesung}
\quizAufgabe[true]{For every normal logic program program~$P$ and every partial interpretation~$\langle T,F \rangle$
             there is at least one unfounded set of~$P$ with respect to~$\langle T, F \rangle$.}
\quizAufgabe[true]{Let~$P$ be a normal program and~$\langle T,F\rangle$ be
             a partial interpretation.
             If \nolinebreak a set $U\subseteq\atom{P}$ is unfounded for~$P$ wrt~$\langle T,F \rangle$,
             then $(U\setminus F)$ is also    unfounded for~$P$ wrt~$\langle T,F \rangle$.}
\quizAufgabe[true]{Let~$P$ be a normal program and~$\langle T,F \rangle\sqsubseteq\langle T',F' \rangle$
             be a partial interpretations.
             If \nolinebreak a set $U\subseteq\atom{P}$ is unfounded for~$P$ wrt~$\langle T ,F  \rangle$,
             then $U$ is also                 unfounded for~$P$ wrt~$\langle T',F' \rangle$.}
\quizAufgabe[false]{Let~$P$ be a normal program and~$\langle T,F\rangle$
             be a partial interpretation.
             If \nolinebreak two sets $U_1,U_2\subseteq\atom{P}$ are unfounded for~$P$ wrt~$\langle T,F \rangle$,
             then \mbox{$(U_1\cap U_2)$} is also               unfounded for~$P$ wrt~$\langle T,F \rangle$.}
\quizAufgabe[true]{Let~$P$ be a normal program and~$\langle T,F\rangle$
             be a partial interpretation.
             If \nolinebreak two sets $U_1,U_2\subseteq\atom{P}$ are unfounded for~$P$ wrt~$\langle T,F \rangle$,
             then \mbox{$(U_1\cup U_2)$} is also               unfounded for~$P$ wrt~$\langle T,F \rangle$.}
\quizAufgabe[true]{Let~$P$ be a normal program and $X$ be a set of atoms.
             If \nolinebreak $(\atom{P}\setminus X)$ is an unfounded set of~$P$ wrt~$\langle X,\emptyset \rangle$, 
             then $X$ is a model of $P$.}
%\quizAufgabe[false]{Let~$P$ be a normal program and~$\langle T,F\rangle$
%             be a partial interpretation.
%             The set $\Cn{\{\head{r}\leftarrow\pbody{r}\mid
%                            r\in P,
%                            (\pbody{r}\cap F)\cup(\nbody{r}\cap T)=\emptyset\}}$
%             is unfounded for~$P$ wrt~$\langle T,F \rangle$.}
\quizAufgabe[false]{Let~$P$ be a normal program.
             For every partial interpretation~$\langle T,F \rangle$, if
             $\WiPi{P}\langle T,F \rangle=\langle T',F' \rangle$ then
             $T'\cap F'=\emptyset$.}
%\quizAufgabe[true]{Let~$P$ be a normal program and~$X$ be a model of $\Comp{\Pi}$.
%             For every sets $T\subseteq X,F\subseteq (\atom{\Pi}\setminus X)$,
%             if $\langle T',F' \rangle=\FiPi{\Pi}{\langle T,F \rangle}$,
%             then it holds that
%             $T'\subseteq X,F'\subseteq (\atom{\Pi}\setminus\nolinebreak X)$.}
\end{quizLoesung}
\end{Loesung}

%%% Solutions in German

%%%%%%%%%%%%%%%%%%%%%%%%%%%%%%%%%%%%%%%%%%%%%%%%%%%%%%%%%%%%%%%%%%%%%%%%%%%%%%
%%%% L�sungen
%%%
%%%\begin{Loesung}
%%%\begin{quiz}
%%%\quizAufgabe[richtig]{Die leere Menge~$\emptyset$ ist unfundiert,
%%%                      da $\head{r}\notin\emptyset$ f\"ur jede Regel
%%%                      $r\in P$.}
%%%% \quizAufgabe[richtig]{Wenn eine Menge $U\subseteq\atom{P}$ f\"ur $\langle T,F\rangle$ 
%%%%                       unfundiert ist, dann erf\"ullt
%%%%                       jede Regel $r\in P$ mindestens eine der folgenden drei
%%%%                       Bedingungen:
%%%%                       \begin{enumerate}\addtolength{\itemsep}{-3mm}
%%%%                       \item $\head{r}\notin U$
%%%%                       \item $(\pbody{r}\cap F)\cup(\nbody{r}\cap T)\neq\emptyset$
%%%%                       \item $\pbody{r}\cap U\neq\emptyset$
%%%%                       \end{enumerate}
%%%%                       Wenn die erste Bedingung zutrifft, dann gilt $\head{r}\notin(U\setminus F)$,
%%%%                       und die zweite Bedingung ist unabh\"angig von $U$ bzw.\ $(U\setminus F)$.
%%%%                       Weiterhin folgt aus der dritten Bedingung,
%%%%                       dass $\pbody{r}\cap (U\setminus F)\neq\emptyset$ oder 
%%%%                       $\pbody{r}\cap (U\cap F)\neq\emptyset$, d.h.\
%%%%                       $\pbody{r}\cap (U\setminus F)\neq\emptyset$ oder 
%%%%                       $\pbody{r}\cap F\neq\nolinebreak\emptyset$ gilt.
%%%%                       Wir haben somit gezeigt, dass mindestens eine der folgenden Bedingungen
%%%%                       f\"ur jede Regel $r\in P$ zutrifft, wenn $U$ f\"ur $\langle T,F\rangle$ 
%%%%                       unfundiert ist:
%%%%                       \begin{enumerate}\addtolength{\itemsep}{-3mm}\renewcommand{\labelenumi}{\arabic{enumi}\rlap{$^\ast$}.}
%%%%                       \item $\head{r}\notin (U\setminus F)$\renewcommand{\labelenumi}{\arabic{enumi}.}
%%%%                       \item $(\pbody{r}\cap F)\cup(\nbody{r}\cap T)\neq\emptyset$\renewcommand{\labelenumi}{\arabic{enumi}\rlap{$^\ast$}.}
%%%%                       \item $\pbody{r}\cap (U\setminus F)\neq\emptyset$
%%%%                       \end{enumerate}
%%%%                       Das bedeutet, dass $(U\setminus F)$ eine unfundierte Menge f\"ur $\langle T,F\rangle$ ist.}
%%%% \quizAufgabe[richtig]{Falls eine Regel $r\in\Pi$ Bedingung 2\footnotemark[1] 
%%%%                       f\"ur $\langle T,F \rangle$ erf\"ullt,
%%%%                       dann erf\"ullt~$r$ diese auch bez\"uglich $\langle T',F' \rangle$,
%%%%                       da $T\subseteq T'$ und $F\subseteq F'$.
%%%%                       Ob Bedingung 1\footnotemark[2]
%%%%                       oder 3\footnotemark[3]\ gilt (bzw.\ nicht gilt), h\"angt nicht von
%%%%                       $\langle T,F \rangle$ bzw.\ $\langle T',F' \rangle$ ab.
%%%%                       D.h.\ wenn Bedingung 1, 2 oder 3 f\"ur eine Regel $r\in\Pi$
%%%%                       bez\"uglich $\langle T,F \rangle$ gilt,
%%%%                       dann gilt diese Bedingung
%%%%                       auch bez\"uglich $\langle T',F' \rangle$.
%%%%                       Falls $U$ f\"ur $\langle T,F \rangle$ unfundiert ist,                      
%%%%                       dann ist $U$ folglich auch f\"ur $\langle T',F' \rangle$ unfundiert.}
%%%\quizAufgabe[falsch]{Wir geben ein Gegenbeispiel an:
%%%                      Bez\"uglich $P=\{a\leftarrow b,c\}$ sind
%%%                      die Mengen $U_1=\{a,b\}$ und $U_2=\{a,c\}$ unfundiert f\"ur
%%%                      $\langle\emptyset,\emptyset\rangle$,
%%%                      aber $(U_1\cap U_2)=\{a\}$ ist nicht unfundiert f\"ur
%%%                      $\langle\emptyset,\emptyset\rangle$.}
%%%\quizAufgabe[richtig]{Wenn $U_1$ und $U_2$ beide unfundiert f\"ur $\langle T,F \rangle$ sind,
%%%                      muss f\"ur jede Regel $r\in P$,
%%%                      sodass $\head{r}\in(U_1\cup U_2)$ und $\pbody{r}\cap(U_1\cup U_2)=\emptyset$,
%%%                      $(\pbody{r}\cap F)\cup(\nbody{r}\cap T)\neq\emptyset$ gelten.
%%%                      Das bedeutet, dass $(U_1\cup U_2)$ unfundiert f\"ur $\langle T,F \rangle$ ist.}
%%%\quizAufgabe[richtig]{Die folgenden Aussagen sind \"aquivalent zueinander:
%%%                      \begin{itemize}\addtolength{\itemsep}{-3mm}
%%%                      \item Die Menge $X$ von Atomen ist ein Modell von $P$.
%%%                      \item F\"ur jede Regel $r\in P$ gilt mindestens eine der Bedingungen                           
%%%                            $\head{r}\in X$, $\pbody{r}\not\subseteq X$ oder $\nbody{r}\cap X\neq\emptyset$.
%%%                      \item F\"ur jede Regel $r\in P$ gilt mindestens eine der Bedingungen                           
%%%                            $\head{r}\notin (\atom{P}\setminus X)$, $\pbody{r}\cap (\atom{P}\setminus X)\neq\emptyset$ oder $\nbody{r}\cap X\neq\emptyset$.
%%%                      \item Die Menge $(\atom{P}\setminus X)$ ist unfundiert f\"ur % die partielle Interpretation
%%%                            $\langle X,\emptyset \rangle$.
%%%                      \end{itemize}}
%%%\quizAufgabe[falsch]{Sei
%%%                     $X=\Cn{\{\head{r}\leftarrow\pbody{r}\mid
%%%                              r\in P,
%%%                              (\pbody{r}\cap F)\cup(\nbody{r}\cap T)=\emptyset\}}$.
%%%                     Falls $X$
%%%                     nicht leer ist, ist f\"ur jede Menge $U\subseteq\atom{P}$,
%%%                     sodass $U\cap X\neq\emptyset$,
%%%                     das Komplement $\atom{P}\setminus U$ kein Modell von 
%%%                     $\{\head{r}\leftarrow\pbody{r}\mid
%%%                              r\in P,
%%%                              (\pbody{r}\cap F)\cup(\nbody{r}\cap T)=\emptyset\}$.
%%%                     Das bedeutet,
%%%                     dass es eine Regel $r\in P$ gibt,
%%%                     f\"ur die $\head{r}\in U$,
%%%                     $(\pbody{r}\cap F)\cup(\nbody{r}\cap T)=\emptyset$ und
%%%                     $\pbody{r}\subseteq \atom{P}\setminus U$, d.h.\ $\pbody{r}\cap U=\emptyset$, gilt.
%%%                     Folglich sind alle Mengen $U\subseteq\atom{P}$,
%%%                     sodass $U\cap X\neq\emptyset$, insbesondere $X$ selbst,
%%%                     nicht unfundiert f\"ur $\langle T,F \rangle$.}
%%%\quizAufgabe[falsch]{Gegenbeispiel: F\"ur $P=\{a\leftarrow a\}$ und
%%%                     $\langle T,F \rangle=\langle \{a\},\emptyset \rangle$
%%%                     gilt $\WiPi{P}\langle \{a\},\emptyset \rangle=\langle \{a\},\{a\} \rangle$.}
%%%% \quizAufgabe[richtig]{%
%%%%   F\"ur jedes Atom $x\in T'$ gibt es (mindestens) eine Regel $r\in\Pi$,
%%%%   sodass $x=\head{r}$, $\pbody{r}\subseteq T\subseteq X$ und
%%%%   $\nbody{r}\subseteq F\subseteq (\atom{\Pi}\setminus X)$,
%%%%   d.h.
%%%%   $X\models (
%%%%     \mbox{$\bigwedge$}_{p \in \pbody{r}} p \wedge
%%%%     \mbox{$\bigwedge$}_{q \in \nbody{r}} \neg q)
%%%%   $
%%%%   gilt.
%%%%   Da $X$ ein Modell von $\Comp{\Pi}$ ist,
%%%%   folgt daraus $x\in X$.\newline
%%%%   F\"ur jedes Atom $x\in F'$ und jede Regel $r\in\Pi$,
%%%%   sodass $x=\head{r}$, 
%%%%   gilt $\pbody{r}\cap F\subseteq\pbody{r}\cap (\atom{\Pi}\setminus X)\neq\emptyset$ oder
%%%%   $\nbody{r}\cap T\subseteq\nbody{r}\cap X\neq\emptyset$,
%%%%   d.h.
%%%%   $X\not\models (
%%%%     \mbox{$\bigwedge$}_{p \in \pbody{r}} p \wedge
%%%%     \mbox{$\bigwedge$}_{q \in \nbody{r}} \neg q)
%%%%   $
%%%%   gilt.
%%%%   Da $X$ ein Modell von $\Comp{\Pi}$ ist,
%%%%   folgt daraus $x\notin X$.
%%%% }
%%%\end{quiz}
%%%% \footnotetext[1]{$(\pbody{r}\cap F)\cup(\nbody{r}\cap T)\neq\emptyset$}
%%%% \footnotetext[2]{$\head{r}\notin U$}
%%%% \footnotetext[3]{$\pbody{r}\cap U\neq\emptyset$}
%%%\end{Loesung}
%%%
